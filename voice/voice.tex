%%%%%%%%%%%%%%%%%%%%%%%%%%%%%%%%%%%%%%%%%
%  My documentation report
%  Objetive: Explain what I did and how, so someone can continue with the investigation
%
% Important note:
% Chapter heading images should have a 2:1 width:height ratio,
% e.g. 920px width and 460px height.
%
%%%%%%%%%%%%%%%%%%%%%%%%%%%%%%%%%%%%%%%%%

%----------------------------------------------------------------------------------------
%	PACKAGES AND OTHER DOCUMENT CONFIGURATIONS
%----------------------------------------------------------------------------------------

\documentclass[11pt,fleqn]{book} % Default font size and left-justified equations

\usepackage[top=3cm,bottom=3cm,left=3.2cm,right=3.2cm,headsep=10pt,letterpaper]{geometry} % Page margins
\usepackage{caption}
\usepackage{subcaption}
\usepackage{bbding}

\usepackage{xcolor} % Required for specifying colors by name
\definecolor{ocre}{RGB}{52,177,201} % Define the orange color used for highlighting throughout the book

% Font Settings
\usepackage{avant} % Use the Avantgarde font for headings
%\usepackage{times} % Use the Times font for headings
\usepackage{mathptmx} % Use the Adobe Times Roman as the default text font together with math symbols from the Sym­bol, Chancery and Com­puter Modern fonts

\usepackage{microtype} % Slightly tweak font spacing for aesthetics
\usepackage[utf8]{inputenc} % Required for including letters with accents
\usepackage[T1]{fontenc} % Use 8-bit encoding that has 256 glyphs

\usepackage{graphicx}

\usepackage{ragged2e}
\usepackage{xcolor}

\usepackage[most]{tcolorbox}
\usepackage[firstpage=true]{background}

\newtcolorbox{mytransparentbox}[1][]{%
  coltext=white,
  arc=0pt,
  auto outer arc,
  enhanced jigsaw,
  opacityback=0.6,
  opacityframe=0.0,
  colback=cyan,
  width=\paperwidth,
  boxrule=0pt,
  oversize=9cm,
  #1
}

\newcommand{\misoline}{
\vspace{-1.5em}
\begin{center}
\line(1,0){360}
\end{center}
}

\renewcommand{\thesection}{\arabic{section}}%

\input{structure} % Insert the commands.tex file which contains the majority of the structure behind the template

\begin{document}

%----------------------------------------------------------------------------------------
%	TITLE PAGE
%----------------------------------------------------------------------------------------

\begingroup
\thispagestyle{empty}
\AddToShipoutPicture*{\put(0,0){\includegraphics{cover}}} % Image background
\centering
\vspace*{5cm}
\begin{mytransparentbox}[height=3cm]
\par\normalfont\fontsize{30}{30}\sffamily\selectfont
\centering\textbf{The Voice : La plus belle voix lilloise}\\
\vspace*{0.5cm}
\misoline
\vspace*{-0.3cm}
\par\normalfont\fontsize{25}{25}\sffamily\selectfont
\hspace{0cm}UCECL \hspace{5.5cm}Oct, 2018
\end{mytransparentbox}
\endgroup

%----------------------------------------------------------------------------------------
%	SPONSOR PAGE
%----------------------------------------------------------------------------------------

\newpage
%~\vfill
\thispagestyle{empty}
\begin{center}
\huge{\textsc{Lille : La plus belle voix Franco-Chinoise 2018}}\\
\LARGE{Produit par : UCECL}\\ 
\includegraphics[scale=0.1]{img/avatar.png}% URL
\vspace{1cm}
\misoline
\huge{Nous tenons à remercier nos sponsors :}\\ 

\begin{figure}[h]
  \centering
  \begin{subfigure}[t]{0.25\textwidth}
    \centering\includegraphics[width=\textwidth]{img/airfrance.jpg}
  \end{subfigure}
  \begin{subfigure}[t]{0.25\textwidth}
    \centering\includegraphics[width=\textwidth]{img/cice.jpg}
  \end{subfigure}
  \begin{subfigure}[t]{0.25\textwidth}
    \centering\includegraphics[width=\textwidth]{img/pangniu.jpg}
  \end{subfigure}
  \begin{subfigure}[t]{0.25\textwidth}
    \centering\includegraphics[width=\textwidth]{img/ppost.jpg}
  \end{subfigure}
  \begin{subfigure}[t]{0.25\textwidth}
    \centering\includegraphics[width=\textwidth]{img/printemps.jpg}
  \end{subfigure}
\end{figure}
\end{center}
\normalsize
% Printing/edition date

%----------------------------------------------------------------------------------------
%	TABLE OF CONTENTS
%----------------------------------------------------------------------------------------

%\chapterimage{logo.jpg} % Table of contents heading image

%\pagestyle{empty} % No headers

%\tableofcontents % Print the table of contents itself

%\cleardoublepage % Forces the first chapter to start on an odd page so it's on the right

%\pagestyle{fancy} % Print headers again

%----------------------------------------------------------------------------------------
%	MAIN TEXT
%----------------------------------------------------------------------------------------

\chapterimage{logo.jpg} % Chapter heading image

\chapter*{Appel la plus belle voix 2018}

\section{Contexte}

Suite \`a la r\'eussite de la derni\`ere \'edition de la plus belle voix chinoise \`a Lille, l'Union des chercheurs et \'etudiants chinois \`a Lille (UCECL) lance des appels pour une nouvelle \'edition en mois d'octobre.

Ce petit concours de chant, dont l'id\'ee vient de la fameuse \'emission t\'el\'evis\'ee The Voice, a re\c cu sa r\'eputation parmi les \'etudiant-es chinois-es durant ces derni\`eres ann\'ees. Il permet aux jeunes \'etudiant-es, surtout ceux qui adorent la musique et chanter, de montrer leur capacit\'e hors \'etude. Nous sommes donc ravi de relancer cet \'ev\`enement exceptionnel dans le mois qui vient, et nous souhaitons aller encore plus loin cette fois-ci. Vous apprenez le chinois ? Vous aimez bien la C-pop ? Vous voulez chanter du chinois devant le public ? Alors ce concours est pour vous, venez vite \`a nous rejoindre !

\section{Calendrier provisoire}

Le concours est constitu\'e de 3 phases :

\begin{itemize}
    \item[\Checkmark] \textbf{Auditions publics} (\emph{offline} \& \emph{online}) : du d\'ebut octobre \`a la mi-octobre
    \item[\Checkmark] \textbf{Demi-finale} (\emph{offline}) : fin octobre (nombre estim\'e de participants 30$\sim$40)
    \item[\Checkmark] \textbf{Finale} (\emph{offline}) : fin octobre (nombre estim\'e de participants 10)
    \item[\Checkmark] Lieu pour la demi-finale \& la finale : TBD
\end{itemize}

\section{R\`eglement du concours}

\subsection{Eligibilit\'e}

Sont \'eligibles au concours : tous les \'etudiants sinophones ou les \'etudiants (fran\c cais ou \'etrangers) qui s'appr\^etent \`a chanter en chinois.

\subsection{Modalit\'e d'inscription}

Tous les participants doivent choisir l'une des deux approches suivantes pour participer \`a l'audition public :
\begin{itemize}[label=\textbullet]
    \item Mode \emph{offline} - se prendre un rdv avec le responsable (d\'esign\'e par l'UCECL) dans votre \'etablissement, et puis chanter en live ;
    \item Mode \emph{online} - enregistrer une pi\`ece de video ou audio d'au moins 30 secondes, puis nous envoyer.
\end{itemize}

\subsection{R\`egles g\'en\'erales}

Tous les participants de la demi-finale et la finale se doivent suivre les r\`egles suivantes :
\begin{itemize}[label=\textbullet]
    \item Vous devez pr\'eparer deux pi\`eces de chansons de vos choix : la premi\`ere en 2 mins pour la demi-finale, et la deuxi\`eme en complet \'eventuellement pour la finale ;
    \item Chanter en duo ou en groupe sont autoris\'es, mais le groupe ne doit pas d\'epasser 5 personnes ;
    \item Vous devez nous envoyer l'accompagnement de vos chansons avant une date \`a determiner ;
    \item Vous devez \'egalement mettre l'accompagnement dans une cl\'e USB et l'apporter avec vous le jour J ;
    \item Vous devez arriver 20 mins en avance le jour J, et 30 mins en avance si vous voulez tester le micro ;
    \item Tous les non respects de ces r\`egles pr\'ec\'edentes peuvent se faire exclure du concours tout de suite.
\end{itemize}

\end{document}